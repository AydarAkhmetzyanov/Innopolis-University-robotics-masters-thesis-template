%!TEX root = root.tex

\chapter{Introduction}
\label{chap:intro}
\chaptermark{Optional running chapter heading}

\section{Spacing \& Type}
\label{sec:section}

This is a section. This is a citation without brackets \citen{A}. and this is one with brackets \cite{A}. These are multiple citations: [\citen{A, B, C}]. Here's a reference to a subsection: \ref{sec:subsection}. The body of the text and abstract must be double-spaced except for footnotes or long quotations. Fonts such as Times Roman, Bookman, New Century Schoolbook, Garamond, Palatine, and Courier are acceptable and commonly found on most computers. The same type must be used throughout the body of the text. The font size must be 10 point or larger and footnotes\footnote{This is a footnote.} must be two sizes smaller than the text\footnote{This is another footnote.} but no smaller than eight points. Chapter, section, or other headings should be of a consistent font and size throughout the ETD, as should labels for illustrations, charts, and figures.  

\subsection{Creating a Subsection}
\label{sec:subsection}

\subsubsection{Creating a Subsubsection}

\paragraph{This is a heading level below subsubsection}

And this is a quote: 
%
\begin{quote}
\blindtext
\end{quote}

This is a table:
% currsize is not set in the long table environment, so we need to set it before we set it up.
\makeatletter
\let\@currsize\normalsize
\makeatother

% tabular environments are set to be single-spaced in the  thesis class,  but long tables do not use tabular
% to get around this, set the spacing to single spacing at the start of the long table environment, and set it back to double-spacing at the end of it

\begin{longtable}{cc}
\caption[This is the title I want to appear in the List of Tables]{This is a caption.} \label{tab:pfams} \\
\hline
A & B \\
\hline
\endfirsthead
\multicolumn{2}{@{}l}{\textbf{Table \thetable} \ldots continued} \\
\hline
A & B \\
\hline
\endhead
a1 & b1 \\
a2 & b2 \\
a3 & b3 \\
a4 & b4 \\
\hline
\end{longtable}


The package ``upgreek'' allows us to use non-italicized lower-case greek letters. See for yourself: $\upbeta$, $\bm\upbeta$, $\beta$, $\bm\beta$. Next is a numbered equation:
\begin{align}
\label{eq:name}
\|\bm{X}\|_{2,1}={\underbrace{\sum_{j=1}^nf_j(\bm{X})}_{\text{convex}}}=\sum_{j=1}^n\|\bm{X}_{.,j}\|_2
\end{align}
The reference to equation (\ref{eq:name}) is clickable. 

\section[Theorems, Corollaries, Lemmas, Proofs, Remarks, Definitions\\ and Examples]{Theorems, Corollaries, Lemmas,\\ Proofs, Remarks, Definitions,\\ and Examples}

\begin{theorem}
\label{thm:onlytheorem}
\blindtext
\end{theorem}

\begin{proof}
I'm a (very short) proof.
\end{proof}

\begin{lemma}
I'm a lemma.
\end{lemma}

\begin{corollary}
I include a reference to Thm. \ref{thm:onlytheorem}.
\end{corollary}

\begin{proposition}
I'm a proposition.
\end{proposition}

\begin{remark}
I'm a remark. 
\end{remark}

\begin{definition}
I'm a definition. I'm a definition. I'm a definition. I'm a definition. I'm a definition. I'm a definition. I'm a definition. I'm a definition. I'm a definition. I'm a definition. I'm a definition. 
\end{definition}

\begin{example}
I'm an example.
\end{example}


\section[Optional table of contents heading]{Section with\\ linebreaks in\\the
name}


\Blindtext[2]




